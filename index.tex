% Options for packages loaded elsewhere
\PassOptionsToPackage{unicode}{hyperref}
\PassOptionsToPackage{hyphens}{url}
\PassOptionsToPackage{dvipsnames,svgnames,x11names}{xcolor}
%
\documentclass[
  letterpaper,
  DIV=11,
  numbers=noendperiod]{scrreprt}

\usepackage{amsmath,amssymb}
\usepackage{iftex}
\ifPDFTeX
  \usepackage[T1]{fontenc}
  \usepackage[utf8]{inputenc}
  \usepackage{textcomp} % provide euro and other symbols
\else % if luatex or xetex
  \usepackage{unicode-math}
  \defaultfontfeatures{Scale=MatchLowercase}
  \defaultfontfeatures[\rmfamily]{Ligatures=TeX,Scale=1}
\fi
\usepackage{lmodern}
\ifPDFTeX\else  
    % xetex/luatex font selection
\fi
% Use upquote if available, for straight quotes in verbatim environments
\IfFileExists{upquote.sty}{\usepackage{upquote}}{}
\IfFileExists{microtype.sty}{% use microtype if available
  \usepackage[]{microtype}
  \UseMicrotypeSet[protrusion]{basicmath} % disable protrusion for tt fonts
}{}
\makeatletter
\@ifundefined{KOMAClassName}{% if non-KOMA class
  \IfFileExists{parskip.sty}{%
    \usepackage{parskip}
  }{% else
    \setlength{\parindent}{0pt}
    \setlength{\parskip}{6pt plus 2pt minus 1pt}}
}{% if KOMA class
  \KOMAoptions{parskip=half}}
\makeatother
\usepackage{xcolor}
\setlength{\emergencystretch}{3em} % prevent overfull lines
\setcounter{secnumdepth}{5}
% Make \paragraph and \subparagraph free-standing
\makeatletter
\ifx\paragraph\undefined\else
  \let\oldparagraph\paragraph
  \renewcommand{\paragraph}{
    \@ifstar
      \xxxParagraphStar
      \xxxParagraphNoStar
  }
  \newcommand{\xxxParagraphStar}[1]{\oldparagraph*{#1}\mbox{}}
  \newcommand{\xxxParagraphNoStar}[1]{\oldparagraph{#1}\mbox{}}
\fi
\ifx\subparagraph\undefined\else
  \let\oldsubparagraph\subparagraph
  \renewcommand{\subparagraph}{
    \@ifstar
      \xxxSubParagraphStar
      \xxxSubParagraphNoStar
  }
  \newcommand{\xxxSubParagraphStar}[1]{\oldsubparagraph*{#1}\mbox{}}
  \newcommand{\xxxSubParagraphNoStar}[1]{\oldsubparagraph{#1}\mbox{}}
\fi
\makeatother

\usepackage{color}
\usepackage{fancyvrb}
\newcommand{\VerbBar}{|}
\newcommand{\VERB}{\Verb[commandchars=\\\{\}]}
\DefineVerbatimEnvironment{Highlighting}{Verbatim}{commandchars=\\\{\}}
% Add ',fontsize=\small' for more characters per line
\usepackage{framed}
\definecolor{shadecolor}{RGB}{241,243,245}
\newenvironment{Shaded}{\begin{snugshade}}{\end{snugshade}}
\newcommand{\AlertTok}[1]{\textcolor[rgb]{0.68,0.00,0.00}{#1}}
\newcommand{\AnnotationTok}[1]{\textcolor[rgb]{0.37,0.37,0.37}{#1}}
\newcommand{\AttributeTok}[1]{\textcolor[rgb]{0.40,0.45,0.13}{#1}}
\newcommand{\BaseNTok}[1]{\textcolor[rgb]{0.68,0.00,0.00}{#1}}
\newcommand{\BuiltInTok}[1]{\textcolor[rgb]{0.00,0.23,0.31}{#1}}
\newcommand{\CharTok}[1]{\textcolor[rgb]{0.13,0.47,0.30}{#1}}
\newcommand{\CommentTok}[1]{\textcolor[rgb]{0.37,0.37,0.37}{#1}}
\newcommand{\CommentVarTok}[1]{\textcolor[rgb]{0.37,0.37,0.37}{\textit{#1}}}
\newcommand{\ConstantTok}[1]{\textcolor[rgb]{0.56,0.35,0.01}{#1}}
\newcommand{\ControlFlowTok}[1]{\textcolor[rgb]{0.00,0.23,0.31}{\textbf{#1}}}
\newcommand{\DataTypeTok}[1]{\textcolor[rgb]{0.68,0.00,0.00}{#1}}
\newcommand{\DecValTok}[1]{\textcolor[rgb]{0.68,0.00,0.00}{#1}}
\newcommand{\DocumentationTok}[1]{\textcolor[rgb]{0.37,0.37,0.37}{\textit{#1}}}
\newcommand{\ErrorTok}[1]{\textcolor[rgb]{0.68,0.00,0.00}{#1}}
\newcommand{\ExtensionTok}[1]{\textcolor[rgb]{0.00,0.23,0.31}{#1}}
\newcommand{\FloatTok}[1]{\textcolor[rgb]{0.68,0.00,0.00}{#1}}
\newcommand{\FunctionTok}[1]{\textcolor[rgb]{0.28,0.35,0.67}{#1}}
\newcommand{\ImportTok}[1]{\textcolor[rgb]{0.00,0.46,0.62}{#1}}
\newcommand{\InformationTok}[1]{\textcolor[rgb]{0.37,0.37,0.37}{#1}}
\newcommand{\KeywordTok}[1]{\textcolor[rgb]{0.00,0.23,0.31}{\textbf{#1}}}
\newcommand{\NormalTok}[1]{\textcolor[rgb]{0.00,0.23,0.31}{#1}}
\newcommand{\OperatorTok}[1]{\textcolor[rgb]{0.37,0.37,0.37}{#1}}
\newcommand{\OtherTok}[1]{\textcolor[rgb]{0.00,0.23,0.31}{#1}}
\newcommand{\PreprocessorTok}[1]{\textcolor[rgb]{0.68,0.00,0.00}{#1}}
\newcommand{\RegionMarkerTok}[1]{\textcolor[rgb]{0.00,0.23,0.31}{#1}}
\newcommand{\SpecialCharTok}[1]{\textcolor[rgb]{0.37,0.37,0.37}{#1}}
\newcommand{\SpecialStringTok}[1]{\textcolor[rgb]{0.13,0.47,0.30}{#1}}
\newcommand{\StringTok}[1]{\textcolor[rgb]{0.13,0.47,0.30}{#1}}
\newcommand{\VariableTok}[1]{\textcolor[rgb]{0.07,0.07,0.07}{#1}}
\newcommand{\VerbatimStringTok}[1]{\textcolor[rgb]{0.13,0.47,0.30}{#1}}
\newcommand{\WarningTok}[1]{\textcolor[rgb]{0.37,0.37,0.37}{\textit{#1}}}

\providecommand{\tightlist}{%
  \setlength{\itemsep}{0pt}\setlength{\parskip}{0pt}}\usepackage{longtable,booktabs,array}
\usepackage{calc} % for calculating minipage widths
% Correct order of tables after \paragraph or \subparagraph
\usepackage{etoolbox}
\makeatletter
\patchcmd\longtable{\par}{\if@noskipsec\mbox{}\fi\par}{}{}
\makeatother
% Allow footnotes in longtable head/foot
\IfFileExists{footnotehyper.sty}{\usepackage{footnotehyper}}{\usepackage{footnote}}
\makesavenoteenv{longtable}
\usepackage{graphicx}
\makeatletter
\def\maxwidth{\ifdim\Gin@nat@width>\linewidth\linewidth\else\Gin@nat@width\fi}
\def\maxheight{\ifdim\Gin@nat@height>\textheight\textheight\else\Gin@nat@height\fi}
\makeatother
% Scale images if necessary, so that they will not overflow the page
% margins by default, and it is still possible to overwrite the defaults
% using explicit options in \includegraphics[width, height, ...]{}
\setkeys{Gin}{width=\maxwidth,height=\maxheight,keepaspectratio}
% Set default figure placement to htbp
\makeatletter
\def\fps@figure{htbp}
\makeatother
% definitions for citeproc citations
\NewDocumentCommand\citeproctext{}{}
\NewDocumentCommand\citeproc{mm}{%
  \begingroup\def\citeproctext{#2}\cite{#1}\endgroup}
\makeatletter
 % allow citations to break across lines
 \let\@cite@ofmt\@firstofone
 % avoid brackets around text for \cite:
 \def\@biblabel#1{}
 \def\@cite#1#2{{#1\if@tempswa , #2\fi}}
\makeatother
\newlength{\cslhangindent}
\setlength{\cslhangindent}{1.5em}
\newlength{\csllabelwidth}
\setlength{\csllabelwidth}{3em}
\newenvironment{CSLReferences}[2] % #1 hanging-indent, #2 entry-spacing
 {\begin{list}{}{%
  \setlength{\itemindent}{0pt}
  \setlength{\leftmargin}{0pt}
  \setlength{\parsep}{0pt}
  % turn on hanging indent if param 1 is 1
  \ifodd #1
   \setlength{\leftmargin}{\cslhangindent}
   \setlength{\itemindent}{-1\cslhangindent}
  \fi
  % set entry spacing
  \setlength{\itemsep}{#2\baselineskip}}}
 {\end{list}}
\usepackage{calc}
\newcommand{\CSLBlock}[1]{\hfill\break\parbox[t]{\linewidth}{\strut\ignorespaces#1\strut}}
\newcommand{\CSLLeftMargin}[1]{\parbox[t]{\csllabelwidth}{\strut#1\strut}}
\newcommand{\CSLRightInline}[1]{\parbox[t]{\linewidth - \csllabelwidth}{\strut#1\strut}}
\newcommand{\CSLIndent}[1]{\hspace{\cslhangindent}#1}

\KOMAoption{captions}{tableheading}
\makeatletter
\@ifpackageloaded{tcolorbox}{}{\usepackage[skins,breakable]{tcolorbox}}
\@ifpackageloaded{fontawesome5}{}{\usepackage{fontawesome5}}
\definecolor{quarto-callout-color}{HTML}{909090}
\definecolor{quarto-callout-note-color}{HTML}{0758E5}
\definecolor{quarto-callout-important-color}{HTML}{CC1914}
\definecolor{quarto-callout-warning-color}{HTML}{EB9113}
\definecolor{quarto-callout-tip-color}{HTML}{00A047}
\definecolor{quarto-callout-caution-color}{HTML}{FC5300}
\definecolor{quarto-callout-color-frame}{HTML}{acacac}
\definecolor{quarto-callout-note-color-frame}{HTML}{4582ec}
\definecolor{quarto-callout-important-color-frame}{HTML}{d9534f}
\definecolor{quarto-callout-warning-color-frame}{HTML}{f0ad4e}
\definecolor{quarto-callout-tip-color-frame}{HTML}{02b875}
\definecolor{quarto-callout-caution-color-frame}{HTML}{fd7e14}
\makeatother
\makeatletter
\@ifpackageloaded{bookmark}{}{\usepackage{bookmark}}
\makeatother
\makeatletter
\@ifpackageloaded{caption}{}{\usepackage{caption}}
\AtBeginDocument{%
\ifdefined\contentsname
  \renewcommand*\contentsname{Table of contents}
\else
  \newcommand\contentsname{Table of contents}
\fi
\ifdefined\listfigurename
  \renewcommand*\listfigurename{List of Figures}
\else
  \newcommand\listfigurename{List of Figures}
\fi
\ifdefined\listtablename
  \renewcommand*\listtablename{List of Tables}
\else
  \newcommand\listtablename{List of Tables}
\fi
\ifdefined\figurename
  \renewcommand*\figurename{Figure}
\else
  \newcommand\figurename{Figure}
\fi
\ifdefined\tablename
  \renewcommand*\tablename{Table}
\else
  \newcommand\tablename{Table}
\fi
}
\@ifpackageloaded{float}{}{\usepackage{float}}
\floatstyle{ruled}
\@ifundefined{c@chapter}{\newfloat{codelisting}{h}{lop}}{\newfloat{codelisting}{h}{lop}[chapter]}
\floatname{codelisting}{Listing}
\newcommand*\listoflistings{\listof{codelisting}{List of Listings}}
\makeatother
\makeatletter
\makeatother
\makeatletter
\@ifpackageloaded{caption}{}{\usepackage{caption}}
\@ifpackageloaded{subcaption}{}{\usepackage{subcaption}}
\makeatother

\ifLuaTeX
  \usepackage{selnolig}  % disable illegal ligatures
\fi
\usepackage{bookmark}

\IfFileExists{xurl.sty}{\usepackage{xurl}}{} % add URL line breaks if available
\urlstyle{same} % disable monospaced font for URLs
\hypersetup{
  pdftitle={BM923 Synthetic Biology Workshop},
  pdfauthor={Morgan Feeney},
  colorlinks=true,
  linkcolor={blue},
  filecolor={Maroon},
  citecolor={Blue},
  urlcolor={Blue},
  pdfcreator={LaTeX via pandoc}}


\title{BM923 Synthetic Biology Workshop}
\author{Morgan Feeney}
\date{2026-04-01}

\begin{document}
\maketitle

\renewcommand*\contentsname{Table of contents}
{
\hypersetup{linkcolor=}
\setcounter{tocdepth}{2}
\tableofcontents
}

\bookmarksetup{startatroot}

\chapter*{Preface}\label{preface}
\addcontentsline{toc}{chapter}{Preface}

\markboth{Preface}{Preface}

This is a Quarto book, generated using the
\texttt{sipbs-compbiol-book-template} GitHub template.

The Preface page is intended as a frontispiece with brief introductory
information about the book and, maybe, its authors.

To learn more about Quarto books visit
\url{https://quarto.org/docs/books}.

\bookmarksetup{startatroot}

\chapter{Introduction}\label{introduction}

The Introduction page is intended as a short introduction to the book.

Like most Quarto books, this is a book created from markdown and
executable code.

This kind of book is an example of literate programming - the
intertwining of nicely-formatted text and images, and executable code.
For example, the \texttt{R} code cell below executes and produces output
when the book is compiled:

\begin{Shaded}
\begin{Highlighting}[]
\DecValTok{1} \SpecialCharTok{+} \DecValTok{1}
\end{Highlighting}
\end{Shaded}

\begin{verbatim}
[1] 2
\end{verbatim}

But the \texttt{R} code cell below does not:

\begin{Shaded}
\begin{Highlighting}[]
\FunctionTok{summary}\NormalTok{(cars)}
\end{Highlighting}
\end{Shaded}

See Knuth (1984) for additional discussion of literate programming.

\part{Early Section}

This \texttt{.qmd} file introduces a \texttt{Part} of the Quarto book.
We use the \texttt{\{\#sec-REFERENCE\}} option to make it
cross-referenceable elsewhere in the text, and the
\texttt{\{.unnumbered\}} option to avoid giving it a section number.

\chapter{Early Section Topic}\label{sec-early-topic}

This \texttt{.qmd} file represents some topic-related text. We use the
\texttt{\{\#sec-REFERENCE\}} option to make it cross-referenceable
elsewhere in the text.

\part{Late Section}

This \texttt{.qmd} file introduces a \texttt{Part} of the Quarto book.
We use the \texttt{\{\#sec-REFERENCE\}} option to make it
cross-referenceable elsewhere in the text, and the
\texttt{\{.unnumbered\}} option to avoid giving it a section number.

\chapter{\texorpdfstring{\texttt{R}
Playground}{R Playground}}\label{r-playground}

\begin{Shaded}
\begin{Highlighting}[]
\NormalTok{\#| context: setup}

\NormalTok{\# Download reporter data}
\NormalTok{download.file(\textquotesingle{}https://raw.githubusercontent.com/sipbs{-}compbiol/BM214{-}Workshop{-}3/main/assets/data/reporter\_curves.csv\textquotesingle{}, \textquotesingle{}reporter\_curves.csv\textquotesingle{})}

\NormalTok{library(ggplot2)}
\NormalTok{library(palmerpenguins)}
\NormalTok{library(tidyverse)}
\end{Highlighting}
\end{Shaded}

\section{Introduction}\label{introduction-1}

This page provides a \texttt{WebR} cell for you to use as a playground
to experiment with some example datasets. You can use this page to
explore data management and visualisation in \texttt{R}.

\section{Playground}\label{playground}

\begin{Shaded}
\begin{Highlighting}[]
\NormalTok{\# Use this WebR cell to experiment with some practice biological datasets}
\end{Highlighting}
\end{Shaded}

\section{Things you can do}\label{things-you-can-do}

This \texttt{WebR} instance has three packages installed:

\begin{itemize}
\tightlist
\item
  \texttt{ggplot2}
\item
  \texttt{GGally}
\item
  \texttt{tidyverse}
\item
  \texttt{palmerpenguins}
\end{itemize}

Open the callout boxes below to see some examples you can try in the
code cell above.

\begin{tcolorbox}[enhanced jigsaw, toptitle=1mm, bottomtitle=1mm, opacityback=0, arc=.35mm, breakable, titlerule=0mm, colback=white, colframe=quarto-callout-tip-color-frame, coltitle=black, rightrule=.15mm, title=\textcolor{quarto-callout-tip-color}{\faLightbulb}\hspace{0.5em}{Play with data from a GitHub repository}, bottomrule=.15mm, toprule=.15mm, leftrule=.75mm, colbacktitle=quarto-callout-tip-color!10!white, opacitybacktitle=0.6, left=2mm]

One of our
\href{https://sipbs-compbiol.github.io/BM214-Workshop-3/}{BM214
workshops} involves a \texttt{WebR}-supported interactive exercise
involving simulated reporter curves. We preload this data in the
\texttt{setup} cell (see source code), and you can interact with it
below with the code:

\begin{Shaded}
\begin{Highlighting}[]
\NormalTok{data }\OtherTok{\textless{}{-}} \FunctionTok{read.csv}\NormalTok{(}\StringTok{"reporter\_curves.csv"}\NormalTok{)}
\FunctionTok{glimpse}\NormalTok{(data)}
\end{Highlighting}
\end{Shaded}

\end{tcolorbox}

\begin{tcolorbox}[enhanced jigsaw, toptitle=1mm, bottomtitle=1mm, opacityback=0, arc=.35mm, breakable, titlerule=0mm, colback=white, colframe=quarto-callout-tip-color-frame, coltitle=black, rightrule=.15mm, title=\textcolor{quarto-callout-tip-color}{\faLightbulb}\hspace{0.5em}{Investigate Palmer's Penguins}, bottomrule=.15mm, toprule=.15mm, leftrule=.75mm, colbacktitle=quarto-callout-tip-color!10!white, opacitybacktitle=0.6, left=2mm]

The \texttt{penguins} dataset contains data about three different
species of penguins. Take a look at the format of the dataset:

\begin{Shaded}
\begin{Highlighting}[]
\FunctionTok{glimpse}\NormalTok{(penguins)}
\end{Highlighting}
\end{Shaded}

You'll see there are eight variables, including \texttt{species},
\texttt{weight}, \texttt{sex}, etc. - some of these variables are
\emph{categorical} (i.e.~a category, like \texttt{species}), and others
are \emph{continuous} (i.e.~numerical). You can see a visual overview of
how the data is related using the \texttt{plot()} function:

\begin{Shaded}
\begin{Highlighting}[]
\FunctionTok{plot}\NormalTok{(penguins)}
\end{Highlighting}
\end{Shaded}

We can visualise the number of penguins of each species in a bar chart:

\begin{Shaded}
\begin{Highlighting}[]
\NormalTok{fig }\OtherTok{\textless{}{-}} \FunctionTok{ggplot}\NormalTok{(penguins, }\FunctionTok{aes}\NormalTok{(species, }\AttributeTok{fill=}\NormalTok{species)) }\SpecialCharTok{+}
         \FunctionTok{geom\_bar}\NormalTok{()}
\NormalTok{fig}
\end{Highlighting}
\end{Shaded}

And break this down in a facet plot, by sex:

\begin{Shaded}
\begin{Highlighting}[]
\NormalTok{fig }\OtherTok{\textless{}{-}} \FunctionTok{ggplot}\NormalTok{(penguins, }\FunctionTok{aes}\NormalTok{(species, }\AttributeTok{fill=}\NormalTok{species)) }\SpecialCharTok{+}
         \FunctionTok{geom\_bar}\NormalTok{() }\SpecialCharTok{+}
         \FunctionTok{facet\_wrap}\NormalTok{(}\SpecialCharTok{\textasciitilde{}}\NormalTok{sex)}
\NormalTok{fig}
\end{Highlighting}
\end{Shaded}

We can make a box and whisker plot of penguin body mass by species:

\begin{Shaded}
\begin{Highlighting}[]
\NormalTok{fig }\OtherTok{\textless{}{-}} \FunctionTok{ggplot}\NormalTok{(penguins, }\FunctionTok{aes}\NormalTok{(}\AttributeTok{x=}\NormalTok{species, }\AttributeTok{y=}\NormalTok{body\_mass\_g, }\AttributeTok{fill=}\NormalTok{species)) }\SpecialCharTok{+}
         \FunctionTok{geom\_boxplot}\NormalTok{()}
\NormalTok{fig}
\end{Highlighting}
\end{Shaded}

And plot the body mass for each sex side-by-side

\begin{Shaded}
\begin{Highlighting}[]
\NormalTok{fig }\OtherTok{\textless{}{-}} \FunctionTok{ggplot}\NormalTok{(penguins, }\FunctionTok{aes}\NormalTok{(}\AttributeTok{x=}\NormalTok{species, }\AttributeTok{y=}\NormalTok{body\_mass\_g, }\AttributeTok{fill=}\NormalTok{sex)) }\SpecialCharTok{+}
         \FunctionTok{geom\_boxplot}\NormalTok{()}
\NormalTok{fig}
\end{Highlighting}
\end{Shaded}

We can investigate correlations, such as between body mass and flipper
length:

\begin{Shaded}
\begin{Highlighting}[]
\NormalTok{fig }\OtherTok{\textless{}{-}} \FunctionTok{ggplot}\NormalTok{(penguins, }\FunctionTok{aes}\NormalTok{(}\AttributeTok{x=}\NormalTok{body\_mass\_g, }\AttributeTok{y=}\NormalTok{flipper\_length\_mm)) }\SpecialCharTok{+}
         \FunctionTok{geom\_point}\NormalTok{()}
\NormalTok{fig}
\end{Highlighting}
\end{Shaded}

We can colour datapoints by species:

\begin{Shaded}
\begin{Highlighting}[]
\NormalTok{fig }\OtherTok{\textless{}{-}} \FunctionTok{ggplot}\NormalTok{(penguins, }\FunctionTok{aes}\NormalTok{(}\AttributeTok{x=}\NormalTok{body\_mass\_g, }\AttributeTok{y=}\NormalTok{flipper\_length\_mm, }\AttributeTok{colour=}\NormalTok{species)) }\SpecialCharTok{+}
         \FunctionTok{geom\_point}\NormalTok{()}
\NormalTok{fig}
\end{Highlighting}
\end{Shaded}

And fit a linear regression to each species separately:

\begin{Shaded}
\begin{Highlighting}[]
\NormalTok{fig }\OtherTok{\textless{}{-}} \FunctionTok{ggplot}\NormalTok{(penguins, }\FunctionTok{aes}\NormalTok{(}\AttributeTok{x=}\NormalTok{body\_mass\_g, }\AttributeTok{y=}\NormalTok{flipper\_length\_mm, }\AttributeTok{colour=}\NormalTok{species)) }\SpecialCharTok{+}
         \FunctionTok{geom\_point}\NormalTok{() }\SpecialCharTok{+}
         \FunctionTok{geom\_smooth}\NormalTok{(}\AttributeTok{method=}\StringTok{"lm"}\NormalTok{)}
\NormalTok{fig}
\end{Highlighting}
\end{Shaded}

\end{tcolorbox}

\begin{tcolorbox}[enhanced jigsaw, toptitle=1mm, bottomtitle=1mm, opacityback=0, arc=.35mm, breakable, titlerule=0mm, colback=white, colframe=quarto-callout-note-color-frame, coltitle=black, rightrule=.15mm, title=\textcolor{quarto-callout-note-color}{\faInfo}\hspace{0.5em}{Note}, bottomrule=.15mm, toprule=.15mm, leftrule=.75mm, colbacktitle=quarto-callout-note-color!10!white, opacitybacktitle=0.6, left=2mm]

\texttt{R} comes with a number of example datasets you can practice
with, including:

\begin{itemize}
\tightlist
\item
  \texttt{mtcars}: fuel consumption and other statistic for 32
  automobiles
\item
  \texttt{Titanic}: information on the fate of passengers on the fatal
  maiden voyage of the ocean liner \emph{Titanic}
\end{itemize}

You can see a full list by running the command

\begin{Shaded}
\begin{Highlighting}[]
\FunctionTok{library}\NormalTok{(}\AttributeTok{help =} \StringTok{"datasets"}\NormalTok{)}
\end{Highlighting}
\end{Shaded}

\end{tcolorbox}

\bookmarksetup{startatroot}

\chapter*{References}\label{references}
\addcontentsline{toc}{chapter}{References}

\markboth{References}{References}

\phantomsection\label{refs}
\begin{CSLReferences}{1}{0}
\bibitem[\citeproctext]{ref-knuth84}
Knuth, Donald E. 1984. {``Literate Programming.''} \emph{Comput. J.} 27
(2): 97--111. \url{https://doi.org/10.1093/comjnl/27.2.97}.

\end{CSLReferences}




\end{document}
